% kuleuventheme2poster by Janez Kren, June 2018, janez.kren@kuleuven.be

\documentclass{beamer}
\usepackage[orientation=portrait,size=a0,scale=1.4,debug]{beamerposter}
% BEAMERPOSTER OPTIONS:
%	orientation= portrait / landscape
%	size= a0 / a2 / a3 / a4
%	scale = change the size of text (e.g. 1.4 increases all fonts by factor of 1.4)


\usetheme[kul,white]{kuleuven2poster}
% THEME OPTIONS 
%	for logo:   kul (default) / kulak / lrd
%	background colour:   blue (default) / white
%	background sedes logo:   no (default) / sedes
%  e.g. [lrd,white,sedes]


% USE YOUR BACKGROUND IMAGE (optinal):
%\titlegraphic{ \includegraphics[height=0.9\paperheight]{mybackground.jpg} } %or, depending on the size: [width=\paperwidth]
%\def\backgdopacity{0.25}  % background image opacity fraction (0=transparent, 1=full colour)


\usepackage[utf8]{inputenc}
\usepackage{ragged2e}  % enables \justifying


% INFO 
\title{H.A.L.D.I.S.} %Handy Autonomous Lifeless Drink-fetching Instructable Servant
\author{\large Maxim Aelterman, Jarrit Boons, Paul Leroy, Josse Van Delm, Wout Verschaeren}
\institute{\large KU Leuven De Nayer}
\date{}
 

\begin{document}
\csname beamer@calculateheadfoot\endcsname %recalculate head and foot dimension

\begin{frame}[t,fragile]

%%
%%  TITLE HEADER
%%
\vspace{2.5\baseh} % extra vertical space

\begin{tikzpicture}
\node [text width=\textwidth, align=left, inner sep=0pt, outer sep=0, kul-blue, font=\bfseries\fontsize{0.6\baseh pt}{2}\selectfont] 	
{ \inserttitle } ;
% This will create a title larger than \Huge. Can be replaced with the line below, to make it smaller.
\end{tikzpicture}

%\textcolor{kul-blue}{\bfseries \Huge \inserttitle}



%%
%% BODY
%%
\vspace{.2\baseh}

\begin{columns}[T,totalwidth=\textwidth]
% COLUMN 1
	\begin{column}{.48\textwidth}
	Inleiding ==> uitleggen over de naam
	\end{column}







% COLUMN 2
	\begin{column}{.48\textwidth}
		
	\begin{block}{Title of the bloc}
	Text for generic block
	\end{block}

	\vspace{11pt}
	\begin{exampleblock}{Example block title}
	Text for example block
	\end{exampleblock}

	\vspace{24pt}
	\begin{block}{\bfseries Theme options for \emph{logo}}
		\begin{itemize}
			\item \texttt{kul} \qquad (default, used if no option is specified) \includegraphics[height=.02\paperheight]{graphics/KUL.pdf}
			\item \texttt{kulak} \quad for Campus Kulak Kortrijk \includegraphics[height=.02\paperheight]{graphics/KULAK.pdf}
			\item \texttt{lrd} \qquad for KU Leuven Research \& Development \includegraphics[height=.02\paperheight]{graphics/LRD.png}
			\end{itemize}
	
			\vspace{24pt}
			Example:
			\begin{verbatim}
			\usetheme[kulak]{kuleuven2}
			\end{verbatim}
	\end{block}

	
\vspace{24pt}
	\begin{block}{\bfseries Theme options for \emph{background colour}}

		\begin{itemize}
			\item \texttt{blue} \quad (default, used if no option is specified) \textcolor{kul-secblue}{\Large$\blacksquare$}
			\item \texttt{white} \textcolor{white}{\Large$\blacksquare$}
		\end{itemize}
		
		\vspace{24pt}
		Example:
		\begin{verbatim}
		\usetheme[lrd,white]{kuleuven2}
		\end{verbatim}
	\end{block}
	
	
\vspace{24pt}
	\begin{block}{\bfseries Theme options for \emph{background logo}}
	\begin{itemize}
		\item \texttt{no} \quad (default, used if no option is specified) 
		\item \texttt{sedes} \qquad poster includes \textit{Sedes Sapientiae}:
	
		\begin{figure}
		\flushright
		\includegraphics[height=.1\paperheight]{graphics/sedes.png} 
		\end{figure} 

		\vspace{-100pt}
		Example:
		\begin{verbatim}
		\usetheme[sedes]{kuleuven2}
		\end{verbatim}
	\end{itemize}
	\end{block}
	
	\end{column}





% COLUMN 3+4 (NESTED)
	\begin{column}{.48\textwidth}
	\justifying
	Nunc mi felis, mattis eu nunc sit amet, convallis posuere felis. Pellentesque a ligula arcu. Ut viverra ipsum et sodales malesuada. Pellentesque lobortis luctus quam non tincidunt. Praesent in cursus purus. Etiam turpis enim, feugiat vel est in, mollis congue mauris. Mauris rhoncus, dui sed cursus vehicula, nulla lacus auctor eros, eget laoreet ipsum tortor vitae tellus. Aenean vel ullamcorper dui. Phasellus luctus turpis sed tellus varius molestie. Mauris quis tempor magna, at ullamcorper tortor. In a tincidunt est. Proin imperdiet in nunc nec eleifend. Aenean fermentum finibus nisi, a auctor turpis mollis quis. Praesent molestie porttitor nisi quis tincidunt. Nulla id enim laoreet, dapibus leo vel, convallis ante.
	
	\vspace{1em}
	\begin{columns}[T,totalwidth=\textwidth]
		
	% COLUMN 3
	\begin{column}{.24\textwidth}
	\begin{block}{Text colour}
		\begin{itemize}
			\item black
			\item \textcolor{kul-blue}{KU Leuven primary blue}, \textcolor{kul-secblue}{secondary blue}, and \textcolor{kul-dark}{dark blue}, \textcolor{kul-light}{light blue}
			\item \textcolor{white}{white} $\leftarrow$ white, when background is dark
			\item \textcolor{gray}{50\% gray }, for text and \textcolor{lgray}{5\% gray} for background
		
			\vspace{5mm}
			\item \textcolor{red}{red text colour}, used for \alert{alert text}
		\end{itemize}	\end{block}

		\end{column}
	
	
	% COLUMN 4
	\begin{column}{.48\textwidth}
	
	\vspace{24pt}
	{\bfseries\large More boxes}	
	
		\begin{center}~
		% box1 = kul-blue background and white text
			\begin{beamercolorbox}[wd=.7\textwidth,sep=4pt,center]{box1}
			box1 scheme
			\end{beamercolorbox}
		\end{center}
					
		\begin{center}~
			% box2 = light gray background and kul-blue text
			\begin{beamercolorbox}[wd=.7\textwidth,sep=4pt,center]{box2}	
			box2 scheme  \\
			second line
		\end{beamercolorbox}
		\end{center}
					
		\begin{center}~
			% box3 = light gray background and black text
			\begin{beamercolorbox}[wd=0.3\textwidth,sep=4pt,right]{box3}
			box3 scheme, aligned right
			\end{beamercolorbox}
			\hspace{11pt}
			
			% box4 = light gray background and red text
			\begin{beamercolorbox}[wd=0.3\textwidth,sep=4pt]{box4}		
			box4 scheme, aligned left
			\end{beamercolorbox}
		\end{center}
					
		\begin{center}~
			% box5 = red background and white text
			\begin{beamercolorbox}[wd=4cm,ht=5cm,sep=4pt,center]{box5}
			box5
			\end{beamercolorbox}
		\end{center}
		\end{column}
		
	\end{columns}	


	% COLUMN 3+4 (NESTED) cont.	
			
		\begin{figure}
		\centering
		\includegraphics[width=0.7\textwidth]{example_figure.pdf}
		\caption{Example graphic}

		\end{figure}		
	
	\end{column}
	
\end{columns}

\end{frame}

\end{document}